\chapter{Towards Universal Software Mining}%
\label{chp:large-scale-mining}

As the field of software mining is constantly growing, notably for the purposes
of assessing and understanding the dynamic evolution of software projects,
there is a clear interest on the part of \gls{ESE} researchers and industrials
in accessing the data stored in the Software Heritage archive.

The main direction of the work presented in this thesis is to make large-scale
analysis of this immense body of data possible in an efficient manner. We limit
our scope to the software artifacts themselves, as studying external project
management metadata (issues, \acrshort{CI}, mailing list discussions, etc.) is
a largely different problem, although relevant to the field in its own way.

For that purpose, we first cataloged different use cases by looking at
software mining studies and qualitatively assessing which types of data they
were extracting for their own purposes. This gives us a general idea of the
queries researchers would want to run on the archive, given the opportunity to
do so.

Of course, we cannot limit ourselves to use cases we find in the literature,
as the scopes of past studies are endogenous to the infrastructure researchers
had at their disposal while conducting it. Part of our goal in making Software
Heritage a platform for universal software mining is to make new research
opportunities available by allowing researchers to run studies that were not
possible before. Therefore, in addition to the use cases we find in the
literature, we also want to understand general interests in future research
directions as a way to expand this horizon of possibilities.

\vspace{0.5cm}

This chapter is based on an article~\cite{swh-benevol2018-universal-analysis}
accepted at the 17th Belgium-Netherlands Software Evolution Workshop (BENEVOL
2018).

\section{Cataloging research needs}

\subsection{Literature Review}%
\label{sec:msr-lit-review}

After various preliminary conversations with researchers working in the field
and having a familiarity with the software mining literature, we have
preconceived notions of how studies on software repositories are designed,
which generally happens in a two-fold process.
The first step is to precisely
\emph{select} the software artifacts that are relevant for the study using a
variety of criteria: studies on Java source code will select files ending with
the \texttt{.java} extension, studies on large projects will require a given
number of revisions in the project, etc. Once the artifacts have been selected,
researchers can \emph{analyze} this new corpus by running custom-made tools and
processing algorithms.

To validate this understanding of the literature more rigorously, we
systematically reviewed 54 papers published in the \emph{16th International
Conference on Mining Software Repositories (MSR
2019)}~\cite{DBLP:conf/msr/2019}. In each paper, we sought two important pieces
of information: (1) what were the \emph{selection criteria} used to narrow the
study to work on a specific set of software artifacts, and (2) which types of
\emph{analyzed data} were required for their analyses.

\definecolor{google-green}{HTML}{B7E1CD}
\definecolor{google-yellow}{HTML}{FCE8B2}
\definecolor{google-blue}{HTML}{C6DAFC}
\definecolor{google-red}{HTML}{F4C7C3}
\definecolor{google-grey}{HTML}{F5F5F5}

\begin{figure}[b]
\begin{tikzpicture}[x=1pt,y=1pt,font=\tiny]
  \begin{sankeydiagram}[
    sankey tot length=90pt,%
    sankey tot quantity=54,%
    sankey min radius=15pt,%
    sankey fill/.style={%
      draw,line width=0pt,
      fill,
      google-grey,
    },
    sankey draw/.style={%
      draw=black,
      line width=1pt,
      line cap=round,
      line join=round,
    },
    %sankey debug,
    ]
    \sankeynodestart{54}{0}{p0}{0,100}

    \sankeyadvance{p0}{60pt}
    \sankeyfork{p0}{48/p1,6/ex1}
    \node[left=5pt of p0, align=center]
        {\footnotesize 54 papers\\ \footnotesize from MSR};

    \sankeyadvance{p1}{60pt}
    \sankeyfork{p1}{44/p2,4/ex2}
    \sankeyadvance{p2}{60pt}
    \sankeyfork{p2}{41/p3,3/ex3}
    \sankeyadvance{p3}{60pt}
    \sankeyfork{p3}{36/p4,5/ex4}
    \sankeyadvance{p4}{60pt}
    \sankeyfork{p4}{28/p5,8/ex5}


    {%
        \tikzset{%
            sankey fill/.append style={%
                line width=0pt,
                google-green,
            }
        }
        \sankeyadvance{p5}{60pt}
        \sankeynodeend{28}{0}{p5}{p5}
        \node[right=15pt of p5, align=center]
            {\footnotesize 28 papers\\ \footnotesize reviewed};
    }

    {%
        \tikzset{%
            sankey fill/.append style={%
                line width=0pt,
                google-red,
            }
        }
        \sankeyturn{ex1}{-90}
        \sankeyadvance{ex1}{20pt}
        \sankeynodeend{6}{270}{ex1}{ex1}
        \node[below=15pt of ex1, align=center]
            {6 papers:\\not mining\\software repos};

        \sankeyturn{ex2}{-90}
        \sankeyadvance{ex2}{20pt}
        \sankeynodeend{4}{270}{ex2}{ex2}
        \node[below=15pt of ex2, align=center]
            {4 papers:\\proprietary or\\homemade dataset};

        \sankeyturn{ex3}{-90}
        \sankeyadvance{ex3}{20pt}
        \sankeynodeend{3}{270}{ex3}{ex3}
        \node[below=15pt of ex3, align=center]
            {3 papers:\\no experimental\\methodology};

        \sankeyturn{ex4}{-90}
        \sankeyadvance{ex4}{20pt}
        \sankeynodeend{5}{270}{ex4}{ex4}
        \node[below=15pt of ex4, align=center]
            {5 papers:\\few handpicked\\projects only};

        \sankeyturn{ex5}{-90}
        \sankeyadvance{ex5}{20pt}
        \sankeynodeend{8}{270}{ex5}{ex5}
        \node[below=15pt of ex5, align=center]
            {8 papers:\\irrelevant\\methodology};
    }

  \end{sankeydiagram}
\end{tikzpicture}
\caption{Sankey diagram of the selection process in our literature review.}%
\label{fig:msr-review-sankey}
\end{figure}

Not all articles in our corpus were relevant to improve our understanding of
software mining studies and had to be removed from the review. In particular,
we excluded:

\begin{itemize}
    \setlength\itemsep{0em}
    \item 6 papers which were not actually mining software repositories (tool
        papers, polling study, meta-analyses, etc.);
    \item 4 papers which used software artifacts corpuses that cannot be
        reconstructed from a software archive (proprietary datasets,
        custom-built datasets, manual example generation, etc.);
    \item 3 papers where no experimental methodology was described;
    \item 5 papers which were analyzing a specific set of handpicked
        repositories, and in which the analysis was only relevant for these
        particular repositories;
    \item 8 papers which had an irrelevant methodology (demonstrations of tools
        for building software corpuses, extensive reliance on context such as
        code snippets from development chats or web pages, etc.).
\end{itemize}

In total, we removed 26 papers as shown in the Sankey diagram
in \cref{fig:msr-review-sankey}, and included the 28 remaining papers in our
review.

\subsection{Selection criteria}%
\label{sec:mining-selection-criteria}

This review allowed confirmation of the two steps previously identified as the
general research pattern followed by most empirical studies: criteria-based
selection of a subset of artifacts, followed by analysis of selected artifacts.
\Cref{tab:selection-criteria} shows a detailed tally of the different criteria
we observed in the study designs.

\begin{table}
    \centering
    \caption{Selection criteria found in MSR literature review.}%
    \label{tab:selection-criteria}
    \begin{tabular}{l l c}
        & \textbf{Criteria} & \textbf{Number of studies} \\
         a. & Number of repository stars & 12 \\
         b. & Programming language of repository & 10 \\
         c. & Matching string or pattern in file contents & 5 \\
         d. & Pre-established list of URLs & 4 \\
         e. & File names and extensions & 4 \\
         f. & Natural language used in the project & 2 \\
         g. & Number of commits or releases & 2 \\
         h. & Number of forks & 2 \\
         i. & Random sampling & 2 \\
         j. & Commit dates & 1 \\
         k. & Licenses & 1 \\
         l. & Number of issues or pull-requests & 1 \\
         m. & Number of contributors & 1 \\
         n. & File sizes & 1 \\
         o. & Lines of code in source blobs & 1 \\
    \end{tabular}
\end{table}

% \TODO{visual grouping of the related categories?}

While some of these criteria would be easy to implement in a research platform
based on the \SWH{} archive as a way to select a subset of artifacts for a
given study design, others would require more work. They can be categorized in
several groups, ordered by difficulty of implementation:

\begin{itemize}
    \setlength\itemsep{0em}
\item \emph{Direct Addressing} (crit.\ d): when the study methodology includes
    a pre-established list of artifact identifiers (e.g., repository URLs or
    revision hashes). As long as a mapping between the unique identifiers and
    the objects in the graph is properly maintained, selecting objects based on
    their identifiers should be trivial.

\item \emph{Artifact Properties} (crit.\ e, g, j, n, m): selection is
    often done on properties that are directly stored as part of the artifacts
    themselves: commit authors or dates, file names etc. This data is
    already present in the database and should only require indexes that allow
    to return all objects matching a given property, assuming these indexes
    are sufficiently performant.

\item \emph{External Metadata} (crit.\ a, l): where the selection uses
    metadata that is not part of the software data tracked in the archive
    itself (contextual popularity information such as number of GitHub stars or
    pull requests). While external metadata is out of scope for this thesis, we
    should acknowledge the frequent need for researchers to filter on
    external and contextual metrics, and leave open the possibility of linking
    the objects in the archive with a metadata store.

\item \emph{Derived Metadata} (crit.\ b, f, h, o, k): for filtering on properties
    that are not directly part of the data, but which can be derived from them.
    By running a programming language recognition tool on the file contents, we
    can index blobs by their detected programming language and allow users
    to filter objects on this criterion. This requires additional indexes as
    well as processing pipelines to generate this derived data.  While some
    derived data can be easily generated (e.g., counting the lines of code in
    a file), the process can be more involved for others: filtering on the
    number of file changes introduced in a given commit requires computing the
    difference between the directory subtrees of all successive commit pairs.

\item \emph{Code Search} (crit.\ c): where the selection happens on the contents
    of the files themselves. Finding all files containing a specific string or
    pattern is significantly harder than filtering on other artifact
    properties. As we noted in \cref{sec:swh-infrastructure}, the total
    size of the file contents is around two orders of magnitude larger than the
    rest of the database, dramatically increasing the infrastructure
    requirements of indexing. In addition, the problem of full-text search
    in source code is a notably hard one in this context, as it requires a high
    level of expressiveness to be able to match on specific tokens or API
    uses, and the source code files are written in arbitrary programming
    languages and cannot necessarily be properly parsed to an AST form.
\end{itemize}

Each selection criteria described here defines one specific filter on the
objects, but researchers generally combine these filters together to narrow
down their corpus of study. Therefore, it is not sufficient to simply provide a
way to select the data based on one of these filters; the partial filters must
be able to be joined together relationally to ensure enough expressiveness in
the selection phase.

\subsection{Analyzed Data}

After having selected the relevant objects of study using specific filters and
heuristics, researchers perform the analysis that constitutes the core of their
experimental design. In order to make this possible in Software Heritage, it is
necessary to understand which categories of data need to be made accessible.

\begin{table}
    \centering
    \caption{Types of analyzed data found in MSR literature review.}%
    \label{tab:analyzed-data}
    \begin{tabular}{l l c}
        & \textbf{Data type} & \textbf{Number of studies} \\
         a. & File contents & 15 \\
         b. & File names & 10 \\
         c. & Commit graph and commit properties & 9 \\
         d. & Commit diffs & 7 \\
         e. & Authors and community graph & 5 \\
         f. & Dependency data & 2 \\
    \end{tabular}
\end{table}

In our review of MSR studies, we identified 6 categories of data that were
frequently analyzed in mining studies, shown in \cref{tab:analyzed-data}.
By far the most common type of analyzed artifacts are file contents,
highlighting the clear need to make the source code files easily accessible.
While there are some studies which only rely on the upper layers of the
artifact graph like community detection algorithms, research on the source
files themselves is evidently of foremost interest in software mining.

Studies generally also require basic artifact properties like file names,
commit messages, dates and authors, etc. Those fields should be relatively easy
to provide from the Merkle DAG of the archive. Some studies also depend on the
topology of the subgraphs themselves: commit chains when analyzing software
evolution, or source subtrees when looking at file hierarchies. This is
important to keep in mind, as it means we cannot always simply provide an
unorganized flat stream of artifacts, and need to leave open the possibility of
running algorithms on the graph which links them together.

Here again, studies will tend to use derived data which can be computed from
the data in the archive. The most prominent example is commit diffs, as a lot
of studies look at the \emph{changes} that were introduced by each commit.

Some types of derived data can pose additional challenges. For instance,
computing dependency graphs will involve a different process for each
programming language, or even each package manager. For complex cases such as
this one, our focus should be to make sure all objects required to compute the
derived data can be provided, so that researchers can generate the corpus
themselves.

\section{Functional requirements}%
\label{sec:functional-requirements}

A cursory review of the literature gives us a good understanding of how \emph{current}
software mining studies are designed and the main categories of data they
exploit, but it should not necessarily be our sole guide for the capabilities we
want to provide in a research platform.
Current research directions are said to be \emph{endogenous}
to capabilities offered by mining platforms: one cannot evaluate the needs of
researchers solely by looking at what researchers \emph{do}, as some data
mining questions might be left unexplored not because of a lack of interest but
due to the lack of an exploitable dataset or platform to study them, which is
precisely what we aim to provide.
In consequence, while the current literature reflects the state of what is
currently achievable for researchers, one of our goals is to open new
possibilities by leveraging the unique properties of the archive: notably its
canonical format, diversity, and universality.

To that end, the efforts in building a mining platform in Software Heritage
should primarily focus on capabilities that would be uniquely impactful on the
state of the art when applied to this particular software development corpus.
Mining studies on handcrafted datasets, small number of repositories or
using metadata more easily accessible in other platforms (e.g., dependency
graphs, call graphs), while important to the field, might not be best
suited for Software Heritage, at least in its first iterations. As a general
rule of thumb, a major strength of running experiments on the archive is that
the universality of the corpus can allow researchers to analyze public source
code exhaustively. This benefit is mostly present when the processing step can
largely be automated; studies requiring a lot of manual curation can leverage
the archive only to a minor extent and will not be our main focus.

As an initial target for the research platform, we identified five main
categories of research requirements for capabilities that we want the platform
to have:

\textbf{Content access}. One of the most common requests is to obtain a set of
file contents stored in the archive based on some specific criteria. Those
requests are usually made for the purpose of analyzing the code itself: code
patterns recognition, language detection, static analysis, malware detection,
and so on.  Occasionally, those requests also require some data preprocessing
to be applied to the file contents before the analysis (comment removal, data
or binary strings removal, etc.).

\textbf{Filtering on metadata}. It is generally useful to filter the query
results depending on some criteria on the file metadata. This metadata
can either already be present in the archived repositories (file extensions,
file names, file sizes, directory depth…), or derived from the data (MIME
types, detected programming language, license…). This metadata has to be
precomputed and indexed along with the files.

\textbf{Content search}. The ability to perform full-text search for
specific code fragments or patterns is very useful to focus
computations on the relevant parts of the code, and it requires an up-to-date
full-text search index.

\textbf{History graph}: The ability to access the revision history graph along
with its associated metadata (authors, commit messages, etc.) and being able to
examine the relationships between the different objects in the revision graphs
is imperative to analyze not only the software itself but its evolution through
time. In the context of Software Heritage, these relationships are also
captured across different repositories: forks point to the same ancestor nodes,
directories that were moved from one repository to another point to the same
object, etc.

\textbf{Provenance indexing}: While the software DAG works top-down (the nodes
only point to their children i.e., their content, but never their parents), it
is also sometimes necessary to be able to list the different parents that point
to a specific object. There is a growing body of literature in empirical
software engineering regarding the ability to track the lineage of a software
artifact at various
granularities~\cite{alexandru2019redundancy,godfrey2011determining,german2009code,swh-provenance-emse}.
There are multiple applications for this: find the
possible file names of a file, the different repositories that contain a code
fragment, a directory or a revision, etc. These traversals on the transposed
version of the DAG are known as ``provenance'' queries, as they allow us to
find out the sources of specific objects.

Of course, the platform should be flexible enough for complex queries that
combine multiple of these capabilities together. For instance, it should be
possible to find all file contents referenced from a file with a specific
extension and which contain some specific function name, or to search nodes in
the history graph of a repository while filtering on their associated metadata.

\section{Challenges}

\subsection{Data volume}

Most of the challenges which stand in the way of providing some form of access
to the entire corpus directly stem from the sheer size of the software archive.
In most cases, allowing people to locally retrieve a large chunk of the archive
to perform local computations is very impractical at the Software Heritage
scale, both from a network transfer perspective and for the undue burden it
would cause on local storage requirements.

Handling the file contents of the archive requires a lot of resources and
expertise. The total size of the blobs (currently $\approx 850$\,TiB) require
large amounts of storage capacity, and the blobs cannot easily be stored on a
single machine using consumer-grade storage. The unusual size distribution of
the blobs, whose median size is approximately 3\,KiB, also makes it challenging
to use industry-grade storage solutions, as they generally are not designed to
store a large quantity of very small files. On conventional systems, some
limitations on inode management may apply. Other distributed storage solutions
like Ceph cannot easily handle a large amount of small files (because of the
per-file overhead needed for replication~\cite{dandrimont2018cephml}), and
require some form of custom content packing to take place beforehand.

Compressing the blobs works well to reduce the size with a compression
ratio of $\approx 2$ (estimated on a random sample of about 1\% of the
archive). Further techniques based on ``packing heuristics'' to compress
similar blobs together should be investigated to reduce their size even
more.

The size of the Merkle DAG itself is more reasonable (around 6\,TiB when stored
in the relational database format), but using it efficiently often requires
various indexes, which significantly increases its size on-disk.  Moreover,
some intensive processing on the graph itself could require having it stored
directly in main memory, which is difficult to achieve on standard machines
that have orders of magnitude less RAM\@.

Even if the recipient of the dataset already has the storage capacity and
expertise to handle such a vast amount of data, transferring it through the
network is impractical and expensive. Sending the whole dataset through a
connection with a speed matching the common industry standard of 1~Gbps would
take more than 60 days, assuming the absence of sequential overhead between
fetches.

Of course, some researchers do not want to analyze the entire corpus but rather
a subset of specific repositories, as we observed in \cref{sec:msr-lit-review}.
Even so, this poses another challenge: the price to pay for deduplication is
that all artifacts, even from relatively small repositories, are scattered
around in the archive. Collecting small and sparsely arranged files from
a distributed storage is generally less efficient than mining a locally
available repository that is stored in a very compact and efficient way.

\subsection{Representation mismatch}

Researchers and data scientists usually try their experiments by prototyping on
small sample sizes, before reaching out to experiment on larger datasets.
Doing so, they generally use tools that are fit for specific data
representations, and thus they often expect the data to be presented in
specific formats. One of the challenges of making software analysis accessible
to them is to help them transform the data from a format well-suited for
\emph{archival} to a format suited for large-scale \emph{analysis}.

Files and directories contained in a specific revision are usually expected to
be represented as on-disk filesystem trees, so the children of the
directories can be directly accessed through the primitives of the filesystem.
In the Software Heritage archive, the deduplication requires this to go
through an additional index of the hashes of the directories. The interface
therefore has to provide a utility to ``flatten'' the compact representation
into a more classical directory structure, although doing so systematically
would invalidate the benefits of deduplication.

For the revision graph itself, there is no current standard of representation,
so most of the research experiments thus far have worked on tool-specific
representations (often depending on the version control system used). While
there is value in providing a universal representation for commit-level
software evolution from different sources, it is still important to provide a
data representation that does not stray too far from what those domain-specific
analysis tools usually expect.

\subsection{Provenance mappings}

Providing a way to look up the different places where an object can be found
(i.e., its \emph{provenance}) is a hard problem, because of the combinatorial
explosion of ancestry mappings.
A single file content can be found in thousands if not millions of origins.
There is a difficult balance to find between reducing the size of the
mappings using intermediate objects in the relationship as layers to compress
the volume of edges, and reducing as much as possible the amount of
indirections that require index hits for performance reasons.

Moreover, maintaining this (bottom-up) provenance index is harder than its
top-down counterparts, since there is no way to know all the objects of the
relationship a priori, and thus represent them with an intrinsic hash for
indexing. These mappings will therefore have to be dynamic, meaning a
provenance query for a given software artifact will give different responses as
more snapshots get archived.

\subsection{Repeatability and reproducibility}

An important part of scientific experiments is reproducibility, which is
something to keep in mind when making a very large and
constantly-evolving dataset available for research applications. While
intrinsic hashes guarantee full consistency of the data at the snapshot level,
it might be useful to provide a way to describe the state of the \emph{entire
  archive} at some point in time. If we are able to reconstruct a previous
state of the archive from a timestamp, including this timestamp along
with the experiment methodology will allow it to be repeated on the exact same
dataset as when it was performed for the first time. That way, an experiment
can be \emph{repeated} by performing it on the timestamped state of the
archive, and \emph{reproduced} by performing it on a different dataset.

The usual way to get an intrinsic identifier of a Merkle DAG is simply to hash
the list of its roots. However, it does not work in this case because the graph
is incomplete: the Software Heritage DAG can have a lot of holes (missing
revisions, files, etc.) that can be added or removed without changing the
intrinsic identifier of the nodes, relying solely on those hashes is not
sufficient to reliably obtain a hash of the archive as a whole.

A better way to achieve this would be to use the journal described in
\cref{sec:swh-infrastructure} by adding timestamps to each inserted
object, then read all the objects from the journal up to the archive timestamp.

\subsection{Expressiveness}

Researchers who want to run analyses on the Software Heritage dataset will
perform \emph{queries} on the archive to describe the computations and the part
of the archive they will be run on. Running these queries on the archive will
require an API that can express these different use cases.

The expressive power of the query language determines how easy it is to use the
different data selection features, computation primitives and result
aggregations when running data selection queries on the dataset. The semantics
of the language have to provide ways of representing and combining those
different operations so that the breadth of computations that queries are able
to represent is as wide and generic as possible for the use cases we
identified.

\section{Roadmap}%
\label{sec:roadmap}

\begin{figure}[b]
    \centering
    \begin{tikzpicture}
	\begin{pgfonlayer}{nodelayer}
		\node [style=diagramitem] (0) at (0, 0) {Dataset \\ in public \\ clouds};
		\node [style=diagramitem] (1) at (3.5, 0) {Local \\ dataset \\ mirrors};
		\node [style=diagramitem] (2) at (7, 0) {Collect \\ use cases};
		\node [style=diagramitem] (3) at (0, -2.5) {Elicit \\ Query DSL};
		\node [style=diagramitem] (4) at (3.5, -2.5) {Index \\ derived \\ data};
		\node [style=diagramitem] (5) at (7, -2.5) {In-memory \\ graph};
		\node [style=none] (6) at (7, -1.25) {};
		\node [style=none] (7) at (0, -1.25) {};
	\end{pgfonlayer}
	\begin{pgfonlayer}{edgelayer}
		\draw [style=arrow] (0) to (1);
		\draw [style=arrow] (1) to (2);
		\draw (2) to (6.center);
		\draw (6.center) to (7.center);
		\draw [style=arrow] (7.center) to (3);
		\draw [style=arrow] (3) to (4);
		\draw [style=arrow] (4) to (5);
	\end{pgfonlayer}
\end{tikzpicture}

    % \includegraphics[width=0.55\textwidth]{img/roadmap}
    \caption{Steps towards creating a research platform in Software Heritage.}%
    \label{fig:roadmap}
\end{figure}

A preliminary step of this work is to make the dataset available in some
format suitable for scale-out analysis, so that Software Heritage and other
researchers can try out a few experiments and document their own use cases.
Some public cloud computing providers like Amazon Web Services or Google Cloud
have public dataset programs, on which we can make the Software Heritage
dataset publicly available without bearing the cost of the storage.

While allowing people to run queries directly on a public cloud instance is
well-suited for one-off experiments, it does not always work well for more intensive
use cases. Researchers having access to hardware resources and software engineering
skills might find it more cost-efficient to run their experiments on a local
copy of the archive. As we build a pipeline to keep the Software Heritage
public datasets up-to-date, we need to provide a way for researchers to have
their own local copy of the dataset for in-house processing.

Once the dataset is available in some format for people to run queries on it,
we will be able to collect more real-world use cases as a way to improve our
understanding of researchers' needs, which in turn guides the platform design:
what are the types of queries that scientists want to run? What are the data
and metadata filters that they need?  What is the sort of information that is
the most often retrieved from the data model? Answers to these questions can
deepen our understanding of the use cases of researchers, and eventually work
out the best ways for them to query the archive in a sufficiently expressive
way, whether through low-level APIs or more abstract domain specific languages.

Real-world use cases also exhibit patterns of access to data derived from
the dataset: diffs between revisions, branching and merging histories,
etc. Isolating this ``derived data'' to index it in the dataset would also be
useful to the platform, as it would significantly improve the performance of
computations on these common use cases.

While the cost of disk access for the file contents cannot be avoided, another
interesting research direction performance-wise will be to store as much of the
history graph as possible directly in memory. This should enable efficient
querying of the complex structures that constitute the archive graph and allow
us to analyze it in more detail.

These different steps, summarized as a roadmap in \cref{fig:roadmap}, together
form a useful overview of the work required to implement a large-scale research
platform for universal software mining. As this thesis inscribes itself in this
long-term plan, these steps will be discussed in more detail in the following
chapters.
