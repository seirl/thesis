\chapter{Conclusion}

This thesis retraces my journey of trying to make an immense corpus, the
entire \emph{graph of public software development}, accessible for software
mining research. This endeavor proved to be challenging. The sheer scale of the
corpus, a graph of more than 20 billion nodes and 200 billion edges, precludes
the use of most conventional approaches for data processing, and necessitates
state-of-the art techniques in big data and graph analysis.
While venturing to organize this development graph in a way that could be
empirically studied, I encountered and tackled challenging issues in both the
technical and research aspects of my work, which I described extensively in
this thesis. In this final chapter, I summarize my main academic contributions
to the fields of empirical software engineering and software mining, and
discuss open questions and future research directions.

\section{Summary of academic contributions}

I have made several contributions to the scientific literature that I describe
in this thesis, taking the form of novel research questions, techniques and
empirical studies. These contributions were, to the best of my knowledge, never
studied to this extent in the extant literature.

\paragraph*{Contextualization}

The first original contribution in my work stems from an inquiry into the
extent to which making a large corpus of software developement data accessible
to researchers can be the most useful to software mining research. For this
purpose, I realized an extensive literature review
(\cref{chp:introduction,sec:msr-lit-review}) which helped determine the most
effective ways in which making the Software Heritage data accessible to
researchers can be of use to future software mining studies
(\cref{sec:functional-requirements}). From this, I derived a concrete roadmap
to create a software mining platform for the Software Heritage archive
(\cref{sec:roadmap}).

\paragraph*{Small-scale artifact processing}

I presented the designs and implementations for two techniques to access and
process small to medium sets of software artifacts (up the order of tens of
thousands) from a large remote repository of artifacts, one based on preparing
bundles of artifacts on the server-side (\cref{chp:vault}) and another using a
virtual filesystem to provide lazy access to objects (\cref{chp:fuse}). The
latter is, as far as I know, the first attempt to create a virtual filesystem
for a universal corpus of source code and software development.

\paragraph*{Scale-out processing of the relational graph}

As described in \cref{chp:graph-dataset}, I have made available several exports
of the graph of development history as an exploitable dataset in a relational
format, along with a reusable pipeline to produce future versions of this
dataset. I have shown how this dataset can be leveraged with \gls{OLAP}
platforms by answering concrete examples of research questions on the entire
corpus. I have also made the dataset available on public clouds in a way that
can be queried by researchers with minimal setup, further reducing the friction
of analyzing the full graph of development history. The usability of these
graph exports for software mining studies were further demonstrated by making
it the object of study in the MSR 2020 mining
challenge~\cite{msr-2020-challenge}.
While other databases of software development history were already accessible
in a similar manner (\cite{GHTorrent,web:github-activity-data,mockus2019woc}),
this is the first time such a vast corpus of deduplicated software artifacts
archived from such a variety of different sources was made available for study,
in a way that is demonstrably exploitable for practical uses.

\paragraph*{Graph compression and exploitation}

In \cref{chp:graph-compression} I introduce graph compression as a way to
fit the entire graph of software development in the RAM of a single machine,
which makes it possible to run complex graph algorithms that are not
necessarily easy to parallelize on large clusters. While the techniques of
graph compression are not new, applying them to the deduplicated graph of
software development is a novel approach in the field of software mining: all
previous works in the literature relied on scale-out approaches for large
datasets.  Graph compression opens new possibilities for systematic empirical
analyses of large graphs of software development artifacts. These research
opportunities are further broadened by the addition of node and edges
properties to the compressed graph; in \cref{chp:graph-metadata} I describe
methods to store these graph properties compactly and so that they can be
queried efficiently from the compressed graph. In
\cref{chp:graph-exploitation}, I present two ways of querying the compressed
graph and its properties at a high level; one involves a simple traversal API,
while the other relies on graph query languages. I also describe different
techniques to subsample the graph into smaller coherent corpuses, as well as a
way to estimate the size of the obtained subgraphs using empirical
measurements.

\paragraph*{Topological properties}

\Cref{chp:topology} describes how I conducted the first empirical exploratory
study on the intrinsic structure of the graph of software development and its
most salient topological properties. I analyzed the following robust measures
of network topology: degree distributions, connected components, shortest path
lengths and clustering coefficients.
Computing these metrics efficiently was made possible by the graph compression
techniques presented in \cref{chp:graph-compression}, and thus I do not expect
that this study could have been realized by the software mining community in
the absence of these contributions. From these metrics, I derived concrete
implications for software mining research by accurately describing the general
form of the different layers in the graph, notably by looking at which
distributions could be considered scale-free.

\paragraph*{Identification of software forks}

Finally, I conducted a second extensive empirical study on the semantic
organization of software projects in groups of forks of varying sizes,
described in \cref{chp:forks}. I introduce two different notions of ``forks''
that can be robustly identified without relying on external forge-level
metadata, instead using shared software artifacts: ``shared commit'' forks and
``shared root'' forks. I quantitatively compare how these definitions of
forking match with the definition based on external forking metadata. In
addition, I empirically analyze how these forks organize in fork networks and
fork cliques of different sizes. I then propose a computationally feasible way
to partition the set of origins in the graph by grouping together repositories
that are all shared commit forks of each other.

\section{Empirical findings and impact on software mining research}

While many of the contributions presented in this thesis are new techniques for
large-scale software mining, I also conducted empirical studies on the graph of
software development, which yielded potentially impactful insights for this
research field. These findings are summarized below as stylized facts.

\paragraph*{Per-layer topology}

While assessing the topological properties of the graph by decomposing it into
different layers, one important finding is that there is a large disparity in
the low-level structure of these layers. As detailed in
\cref{sec:topology-discussion}, the three main layers have dramatically
different shapes, densities and connectivities, and each take up very uneven
shares of the total size of the graph. The properties of the full graph are
largely dominated by those of the filesystem layer, which represents the vast
majority of nodes (90\%) and edges (97\%) in the graph.
This highlights the importance of studying graph layers separately when
assessing the graph properties, as it can have a considerable impact on the
findings.

\paragraph*{Graph connectivity and sharding}

One important empirical finding from the topology analysis is that the graph is
generally well connected due to the deduplication of the software artifacts it
contains. This is particularly true at the level of the filesystem, where 97\%
of all the files and directories in the graph are all reachable from each other
and form a giant connected component. This giant component precludes the use of
naive partition techniques, such as regrouping nodes in small tightly connected
clusters, as a way to ``shard'' the graph on multiple machines in order to
scale-out computations. This high connectivity is resilient, and does not
depend on the existence of a few nodes with large in-degrees; removing these
nodes does little to help partition the giant component.

However, restricting our view to the history layer yields a very different
picture: revision chains are less connected, with the giant component only
encompassing 3\% of the total number of revisions. This indicates that the
history layer can easily be partitioned in reasonably sized groups of connected
clusters, which is impactful for empirical studies looking at distributing the
nodes in the history layer on multiple machines.

\paragraph*{Kurtosis, scale invariance and heavy tails}

Some of the frequency distributions studied in this thesis have heavy tails,
a relatively high kurtosis and could likely be scale-invariant. This has
important implications for empirical software engineering studies using data
filtering and sampling as a way to get a representative set of software
artifacts to analyze. Notably, it means that there is often no obvious rule to
filter out outliers that cross a certain threshold, as they are an integral
part of the distributions' nature. This emphasizes that empirical software
mining studies should be cautious to systematically justify their data
selection methodology, and consider sampling biases potential threats to
validity.

\paragraph*{Fork identification}

A key empirical result of the fork identification study is that relying on
external forge-level metadata as the sole source of information to determine
whether two repositories are forks of one another tends to miss a
significant share of what people intuitively think of as software forks.
By instead using the ``shared commits'' definition of a fork, we identified
around 9\% more forks than when using the GitHub forking information,
suggesting that a non-negligible number of repositories are forked without
using forge forking tools. We expect that this number would be even higher when
including data from other hosting places, as there is no external source of
information to identify forks across different forges. Empirical software
mining studies on software forks aiming to be exhaustive in their coverage and
reduce potential sources of bias should thus consider using fork definitions
based on \emph{shared \gls{VCS} history} rather than trusting forge-level
metadata, using the identification technique based on the compressed graph
described in \cref{chp:forks}.

\section{Future work}

The work presented in this thesis improves the state-of-the-art of large scale
software mining platforms, and by doing so raises a significant number of open
questions and enables new avenues for future research. Below is a summary of
what we consider to be most promising next steps in pursuing this research.

\paragraph*{Incremental graph compression}

By far the most impactful problem that could be tackled is to produce a
functional design for \emph{incremental graph compression}, i.e., having an
updating pipeline constantly maintaining the compressed representation of the
graph up-to-date with the latest software artifact data. Doing so would
considerably reduce the delay between the current state of the archive and the
data stored in the compressed graph, from entire months down to a few hours or
less. This would allow the research platform to answer complex queries on the
archive by using compressed graph algorithms only, and without having to
fallback on slower algorithms for artifacts that were not present at the time
of compression.

Incremental graph compression would be a very challenging endeavour, as most of
the components in the compression pipeline relies on a fixed number of nodes.
For instance, one of the core steps of the process is the use of \gls{MPH}
functions, which rely on a static number of elements and cannot be dynamically
updated. One way to address this problem could be to use \emph{dynamic perfect
hashing} algorithms~\cite{fredman1984storing,dietzfelbinger1994dynamic},
although these have fairly different properties, including the non-contiguity
of their output, which pose their own set of challenges.

\paragraph*{Typed graph compression}

While the \gls{LLP} algorithm (\cite{sec:llp-compression}) for graph reordering
yields achieves impressive compression ratios, we strongly suspect that even
better results could be achieved by leveraging \emph{typed} graph compression,
i.e., compressing all the nodes and edges of each of the six node types in
their own separate graph. This is because there is a higher chance that
copy-lists used in the compressed BVGraph would be more similar between nodes
of the same type, but this locality cannot be exploited with algorithms based
solely on node proximity and clustering.

\paragraph*{Graph querying}

\Cref{sec:graph-querying} describes a simple traversal API for remote graph
querying, and outlines its limitations by showing examples of queries that it
cannot easily answer. It also presents graph query languages as a possible
solution to limited expressiveness, but these were not actually implemented as
an interface to the compressed graph. Doing so could significantly improve the
accessibility of the research platform for more complex experiments. In a
comprehensive platform, this could then further be extended to include
computation primitives on the objects themselves, down to the file contents.
One could imagine writing a single query to get the average number of lines of
codes of all the files in the archive in a path matching
\mintinline{bash}{fr_FR/*.po}.  This would require a comprehensive integration
of all the components of the research platform: graph structure querying,
content fetching and processing, online distributed computing, etc. While this
is a long term goal, this thesis provides the basic building blocks to achieve
such a feat.

\paragraph*{Scheduling predictions}

This thesis presents a work which leverages the Software Heritage archive to
build an analysis platform open to all software mining researchers. Among them,
one of the primary beneficiary of this platform could be the Software Heritage
initiative itself, as analyzing the data in the archive could yield valuable
insights that could be used to fine-tune crawling behavior. One example of this
is for scheduling predictions: the compressed graph augmented with commit
timestamps can efficiently compute the frequency at which a given repository is
updated; feeding that information back to the scheduler could help prioritizing
the crawling of repositories that get updated often, and reduce the crawling
frequency of repositories that do not. This could alleviate the total load on
the archive workers and increase its throughput.

\paragraph*{Graph partitioning techniques}

In \cref{sec:topology-discussion}, we discuss how the high connectivity of the
graph precludes the use of naive graph partitioning techniques to shard the
graph into smaller isolated units, as most of the nodes in the graph are
aggregated in a giant connected component that cannot easily be broken down.
While this remains generally true, if sharding is needed it is still possible
to try and regroup nodes using their locality characteristics to decide how to
partition the graph. In \cref{chp:graph-compression} we use two approaches,
\gls{BFS} and \gls{LLP}, to generate a node ordering which preserves locality.
It would be interesting to partition the graph nodes by splitting these
locality-preserving orders in chunks of equal size. These partitions could then
be benchmarked for sharding efficiency to find out whether parallel graph
querying workloads would require many synchronization points between the
different shards, or just a few. This could be extended to other clustering
techniques that were not investigated in this thesis, such as using Louvain
community detection~\cite{blondel2008fast} to delimitate groups of nodes.

\paragraph*{Derived graphs}

One last interesting research opportunity would be to make accessible data
that can be derived from the graph for software mining research. One such
example is the graph of collaboration in software development, where the nodes
are unique persons and the edges represent whether two authors collaborated on
a common project; this could be generated using a linear traversal on the
compressed graph, and would represent a real-world social network whose
structure has, to our knowledge, not yet been extensively studied in the
social network literature. Another example would be to compute the ``diff
graph'' of development history, where instead of each revision pointing to the
root directory of its source tree, it would instead point to the changes that
happened since the last revision (i.e., new, deleted, modified and renamed
files). This could help answer more research questions efficiently, such as
finding the list of revisions in which a given file was added for the first
time. Constructing this derived graph has been experimented at a reasonably
large scale (10\% of the edges in the graph) as part of an unpublished study
with Wellenzohn et al, delivering promising results.

\section{Publications}

\begin{enumerate}
    \item \bibentry{swh-benevol2018-universal-analysis}
    \item \bibentry{swh-msr2019-dataset}
    \item \bibentry{msr-2020-challenge}
    \item \bibentry{saner-2020-swh-graph}
    \item \bibentry{swh-msr2020-forking}
    \item \bibentry{msr-2020-topology}
    \item \bibentry{swh-2021-swhfs}
\end{enumerate}
