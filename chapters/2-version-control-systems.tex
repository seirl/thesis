\chapter{Version Control Systems}

The work described in this thesis is about organizing the \emph{software
artifacts} in the Software Heritage graph to make them accessible to
researchers. These artifacts are abstract building blocks that represent the
source code trees, development history and hosting data stored in the archive.
In the next chapter, we will look at how the Software Heritage data model is a
graph built on the relationships between software artifacts. However, to better
grasp this model, it is good to first get a better understanding of how
software artifacts are captured in the data model of traditional \gls{VCS}.

This chapter describes a generic model for how data is stored in most
\gls{VCS}; it is very close to Git, the most popular of these systems, while
being abstract enough to be independent of any specific implementation.

\section{A simple code repository model}

\subsection{Files and directories}

The fundamental elements in a source code repository are, naturally, source
code files. Developers tend to organize their code in different logical units
that are then hierarchized in different levels of directories. At the lowest
granularity, programmers write \emph{functions} that perform a specific task or
computation. A \emph{source code file} is generally a collection of
conceptually related functions, put together as a single logical unit. Source
files that define the behavior of a logically distinct component of a program
are also often logically grouped in the same directory, sometimes called a
\emph{module} or \emph{package}.

\begin{figure}
    \centering
    \begin{subfigure}[b]{.40\textwidth}
        \dirtree{%
            .1 /.
                .2 README.txt.
                .2 src.
                    .3 evalexpr.c.
                    .3 parser.
                        .4 ast.c.
                        .4 parser.c.
                        .4 lexer.c.
                .2 tests.
                    .3 eval.c.
                    .3 operands.c.
        }
        \caption{Directory listing}
        \label{fig:vcs-dir-flat}
    \end{subfigure}\hfill
    \begin{subfigure}[b]{.58\textwidth}
        \centering
        \begin{tikzpicture}[scale=2, font=\footnotesize]
	\begin{pgfonlayer}{nodelayer}
		\node [style=directory] (9) at (3, -2) {};
		\node [style=directory] (10) at (1, -2) {};
		\node [style=directory] (14) at (1.5, -3) {};
		\node [style=content] (16) at (1.5, -4) {};
		\node [style=content] (17) at (2.25, -4) {};
		\node [style=content] (18) at (0.75, -4) {};
		\node [style=content] (19) at (0, -3) {};
		\node [style=directory] (20) at (2, -1.5) {};
		\node [style=content] (21) at (2.5, -3) {};
		\node [style=content] (22) at (3.5, -3) {};
	\end{pgfonlayer}
	\begin{pgfonlayer}{edgelayer}
		\draw [style=arrow] (20) to node [above,sloped] {src} (10);
		\draw [style=arrow] (10) to node [above,sloped] {evalexpr.c} (19);
		\draw [style=arrow] (10) to node [above,sloped] {parser} (14);
		\draw [style=arrow] (14) to node [above,sloped] {ast.c} (16);
		\draw [style=arrow] (14) to node [above,sloped] {lexer.c} (17);
		\draw [style=arrow] (14) to node [above,sloped] {parser.c} (18);
		\draw [style=arrow] (20) to node [above,sloped] {tests} (9);
		\draw [style=arrow] (9) to node [above,sloped] {parser.c} (21);
		\draw [style=arrow] (9) to node [above,sloped] {operands.c} (22);
	\end{pgfonlayer}
\end{tikzpicture}

        \caption{Represented as a tree structure}%
        \label{fig:vcs-dir-tree}
    \end{subfigure}
    \caption{Example of a directory hierarchy for a code repository}
    \label{fig:vcs-dir-project}
\end{figure}

The example in \figref{fig:vcs-dir-flat} illustrates a file hierarchy for a
project meant to evaluate simple mathematic expressions. There is always one
root directory containing the entire project, that we represent with a slash.
At the root level, there is a README file, a directory containing C
source code files with some of them organized in a parser module, and a test
directory containing tests also written in C.

The file hierarchy in this example project can naturally be conceptualized as a
tree data structure, by representing the directories and the file contents as
the vertices, with their respective names on the edges. The resulting tree is
shown in \figref{fig:vcs-dir-tree}.

\subsection{Commits}

The concept of \acrlong{VCS} arose from the need of keeping track of the
changes happening in the code. This ability is critically important for
software systems, as it allows potentially erroneous changes to be reverted
without having to remember what the previous behavior used to be. Besides, the
ability to retain institutional knowledge on a codebase by having the
possibility to look up past source code changes was very valuable.

Traditionally, this ``versioning'' could be done by simply copying the software
source tree to a new directory for each new version and keeping the old
versions as archives of the past state of the code. This manual process is
generally tedious and discourages small incremental changes, because of how
convoluted it generally is to create an entire new source tree. It also hinders
collaboration between developers, who have to exchange work in progress source
files and can easily lose track of which version they are working on.

However, this basic concept of keeping track of "snapshots" of how the source
tree used to look like at different points in time was a key design insight in
the development of most modern \gls{VCS}. They introduce the notion of
``commits'', which we can conceptualize as a chain of frozen past states of a
source tree that were recorded at various points in time during development.
