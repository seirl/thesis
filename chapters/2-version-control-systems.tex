\chapter{Version Control Systems}

The work described in this thesis is about organizing the \emph{software
artifacts} in the Software Heritage graph to make them accessible to
researchers. These artifacts are abstract building blocks that represent the
source code trees, development history and hosting data stored in the archive.
In the next chapter, we will look at how the Software Heritage data model is a
graph built on the relationships between software artifacts. However, to better
grasp this model, it is good to first get a better understanding of how
software artifacts are captured in the data model of traditional \gls{VCS}.

This chapter describes a generic model for how data is stored in most
\gls{VCS}; it is very close to Git, the most popular of these systems, while
being abstract enough to be independent of any specific implementation.

\section{Files and directories}


