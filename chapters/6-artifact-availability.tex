\chapter{Artifact Availability}

In the previous chapters, we have outlined the different challenges in making
Software Heritage a universal software mining platform. By studying the needs
of researchers in the fields of empirical software engineering performing
studies on software repositories, it is possible to get a good sense of the
best ways the archive could become useful as a research platform.

However, building the universal platform constitutes a huge endeavour that has
to be done incrementally. As we cannot aim to cover most use-cases overnight,
we should instead prioritize features that are immediately useful to
researchers so that they can start using the building blocks we provide in
their work. Aside from driving up engagement with the research community, this
also helps identifying the main pain points they encounter, which in turn can
guide the roadmap in the steps we take to alleviate them.

The very first step that has to be taken to make the archive usable as a
bare-bones platform is to provide a simple way to retrieve single artifacts or
logical groups of artifacts from our own storage. While this cannot necessarily
make large-scale analysis possible, this could at least be used for some
smaller-scale experiments and prototypes.

For single artifacts, the Software Heritage
API\footnote{\url{https://archive.softwareheritage.org/api/}} already covers
this use-case in some way: using the \gls{SWHID} of an object, it is possible
to retrieve its associated data as stored in the archive. As an example, the
following HTTP GET request can be used to get the revision
\texttt{swh:1:rev:aafb16d69fd30ff58afdd69036a26047f3aebdc6}:

\begin{lstlisting}[language=python,frame=single]
# GET https://archive.softwareheritage.org/api/1/revision/aafb16d69fd30ff58afdd69036a26047f3aebdc6/

{
    "id": "aafb16d69fd30ff58afdd69036a26047f3aebdc6",
    "message": "Merge branch 'master' into pr/584\n",
    "author": {
        "email": "nicolas.dandrimont@crans.org",
        "fullname": "Nicolas Dandrimont <nicolas.dandrimont@crans.org>",
        "name": "Nicolas Dandrimont"
    },
    "date": "2014-08-18T18:18:25+02:00",
    "directory": "9f2e5898e00a66e6ac11033959d7e05b1593353b",
    "metadata": {},
    "parents": [ ... ]
    [ ... ]
}
\end{lstlisting}

\section{The Vault}

\section{Contents}
