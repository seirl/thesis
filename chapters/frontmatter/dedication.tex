\chapter*{Remerciements}

\begin{otherlanguage}{french}
\begin{SingleSpace}

  Je tiens tout d'abord à remercier Stefano Zacchiroli pour avoir encadré cette
  thèse jusqu'à son aboutissement. \emph{Zack} était pour moi le directeur de
  thèse idéal ; il a su être à la fois présent et encourageant, tout en me
  laissant suffisamment de liberté et d'autonomie dans ma recherche. Merci
  également à Roberto Di Cosmo,\footnote{Zack et Roberto sont de vraies
  \emph{rockstars} du logiciel libre : Zack était Debian Project Leader, et
  Roberto a créé la première distribution Linux pour Live CD.} directeur
  de Software Heritage, pour m'avoir proposé cette thèse et avoir toujours
  soutenu mon projet de recherche.
  Ces remerciements s'étendent
  naturellement aux autres chercheurs et ingénieurs de l'équipe de Software
  Heritage qui furent un important soutien à la fois technique et moral pendant
  plus de quatre ans. Merci à (par ordre d'apparition) Nicolas Dandrimont,
  Antoine Dumont, Guillaume Rousseau, Morane Gruenpeter, Antoine
  Lambert,\footnote{Oui, ça fait trois Antoine, bientôt nous dominerons le
  monde.} David Douard, Valentin Lorentz et Vincent Sellier. Merci également à
  Thibault Allançon qui, en l'espace de seulement deux stages, a implémenté une
  infrastructure solide sur laquelle repose une grande partie de ma thèse.

  Je remercie également toute ma famille pour leurs encouragements, notamment
  mes parents qui ont rendu cette aventure possible et m'ont toujours soutenu
  dans mes choix d'orientation et dans mes projets. Merci à ma mère qui a été
  particulièrement attentive durant mes études, et mon père chez qui je partais
  tous les étés me changer les idées dans les montagnes de l'Alta
  Rocca.\footnote{Je reste à ce jour persuadé que Stefano et Roberto pensaient
  recruter un autre Italien lorsqu'ils m'ont pris en thèse, et qu'ils ont été
  déçus d'apprendre que mon nom était en fait corse.}
  Merci à mes frères : Pierre-Louis, Mathieu et François,
  à mon cousin Victor,
  qui est sans doute la première personne à m'avoir initié à
  l'informatique,\footnote{Notre complicité n'a d'égal que notre bêtise : nous
  sommes mus depuis notre plus jeune âge par une inébranlable volonté de tout
  péter, ce à quoi nous nous appliquons avec créativité et détermination.}
  à mes cousines : Amélie, Pauline, Claire et Margot,
  à mes tantes et oncles : Marie-Pascale, Christine,\footnote{Pour
  se motiver mutuellement pendant notre période de rédaction, on a fait la
  course. J'ai fini par rendre ma thèse avant qu'elle rende son HDR, mais à
  seulement dix jours près.}
  Pierre,\footnote{Lui aussi a fait une thèse, mais elle fait 1240
  pages et résume 42 ans de recherche. On ne joue pas dans la même cour.}
  Chantal et Jean-Pierre, à mon grand-père Jean-Louis qui me fascinait
  avec ses points de Lagrange et ses programmes BASIC sur disquette, à Thibault
  Hanin et ses panneaux solaires, et à toute ma famille plus ou moins éloignée.
  Merci également à Marie-Antoinette et Geneviève de prendre régulièrement de
  mes nouvelles.

  Cette thèse aurait sans doute été très différente sans l'aide de plusieurs
  personnes. Je remercie Jill-Jênn Vie\footnote{Monsieur le Président.} qui, en
  plus d'avoir été un ami très proche pendant plus de dix ans, et avec qui je
  partage de nombreuses passions informatiques et culturelles, a également
  trouvé ma thèse,\footnote{Il est allé voir le stand de Software Heritage à
  une convention Open Source, et m'a dégoté un entretien avec Roberto et Zack,
  qui m'ont proposé un sujet de doctorat deux jours plus tard.} puis l'a
  intégralement relue.\footnote{Il aime aussi les zeugmas (et Tomoko).}
  Thanks to Amie Harte for reviewing every single sentence of this thesis and
  being such a great friend.\footnote{She shares my obsession of always being
  right, and is almost too good at it for my taste.} Merci à Théo Zimmermann
  d'avoir relu certains de mes articles et de m'avoir donné de précieux
  conseils. Merci à Mathieu Tarral qui m'a expliqué le langage Cypher.

  Ces quatre années de thèse furent pour moi un véritable plongeon dans
  l'univers de la recherche académique, mais ma première initiation à celle-ci
  date de quelques années auparavant. Merci à Akim Demaille pour m'avoir donné
  goût à la recherche en m'encadrant pendant deux ans au LRDE sur VCSN, un
  projet qui fut pour moi très enrichissant tant d'un point de vue
  scientifique que technique.

  Merci à Ambre Williams, Thibault Suzanne, Nicolas Jeannerod, Victor Lanvin,
  Athénaïs Vaginay, Adrien Fabre, Behrang Shirizadeh, Thomas Hegland et tous
  les doctorants et jeunes docteurs que j'ai eu le privilège de côtoyer pendant
  ma thèse.

  Merci à mes plus ou moins lointains amis avec qui la distance n'est
  bien heureusement que géographique.\footnote{Pour être honnête, tant que
  le lag sur IRC reste raisonnable, ça ne change pas grand chose.} C'est
  toujours un immense plaisir de retrouver Rémi Audebert, Sophie Boissonneau,
  Alexandre Macabiès, Pierre Bourdon,\footnote{Qui, contrairement aux autres
  personnes nommées ici, a eu une influence extrêmement mauvaise sur moi, en me
  conseillant sans cesse d'arrêter ma thèse et de jouer à des jeux vidéo.}
  Valentin Tolmer et Sami Boukortt au détour d'un lac suisse, Paul Hervot dans
  un café strasbourgeois, Antoine Lecubin et Tomoko Kozu dans un
  \emph{onsen} japonais, ou même Sarah Fachada-Dury et Mercedes Haiech dans
  les contrées perdues de Gwendalavir.

  Regarder un mauvais film est, contre toute attente, une excellente façon de
  passer une bonne soirée quand on est bien entouré. Merci à Stéphane Lefebvre,
  Antoine Becquet, Nicolas Allain, Nicolas Salaün, Jennyfer Burlet, Alexandre
  Meunier, Lucie Mader, Maël Le Garrec, Marcelin Dupraz et Audrey Tinney pour
  tout ce bon temps passé à regarder des ninjas, requins mutants et autres
  reptiliens au centre de la Terre.

  Organiser pendant dix ans le concours national d'informatique était une
  formidable aventure, et je tiens à remercier Cyril Amar, Aliona Dangla,
  Mélanie Godard, Garance Gourdel, Céline Baraban, Jérémie Marguerie, Sacha
  Delanoue, Nicolas Hureau, Marin Hannache, Maxime Audouin, Nicolas Baradakis,
  Julien Guertault, Sylvain Laurent, Enguerrand Decorne, Héloïse Vallerio,
  Victor Collod, Kaci Adjou, Arthur Remaud, Éloi Charpentier, Rémi Dupré ainsi
  que tous les autres membres de Prologin avec qui j'ai eu le plaisir de
  travailler, si nombreux que je ne pourrais en dresser une liste exhaustive.

  Enfin, merci à tous mes amis de plusieurs époques avec qui mes relations
  furent aussi diverses qu'enrichissantes : Julian Farella, Romain
  Maurizot, Guillaume Sanchez, Guillaume Sanchez,\footnote{L'autre.} Alexandra
  Kientz, Pierre-Olivier Di Giuseppe, Thomas Barusseau, Adrien Schildknecht,
  Juliette Tisseyre, Lucille Tachet, Anthony Seure, Solène Pichereau, Yoann
  Bourse, Sarah Erika Elvang, Laurent Xu, Anatole Moreau, Arnaud Agbo, Céline
  Chenu, Ange Daumal, Izia Pétillon, Sébastien Pétillon, Mathieu Stefani, Jean
  Dupouy, Alexis Cassaigne, Rémi Labeyrie, Julien Morais, Garance Coggins,
  Juliette Conte et bien d'autres.

\end{SingleSpace}
\end{otherlanguage}
