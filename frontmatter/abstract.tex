\begin{minipage}[t][0.5\textheight][t]{0.96\textwidth}

\section*{Abstract}
\begin{SingleSpace}
    The Software Heritage project is a software archive containing the largest
    public collection of source code files along with their development history,
    in the form of an immense graph of hundreds of billions of edges. In this
    thesis, we present architectural techniques to make this graph available
    for research. We first propose some utilities to access the data at a
    micro-level in a way that is convenient for smaller-scale research.
    To run analyses on the entire archive, we extract a property graph in a
    relational format and evaluate the different ways this data can be
    exploited using in-house and cloud processing services.
    We find that while this approach is well suited to process large amounts of
    flat data in parallel, it has inherent limitations for the highly-recursive
    graph structure of the archive. We propose the use of graph compression as
    way to considerably reduce the memory usage of the graph, allowing us to
    map it entirely in physical memory. We develop a library to run arbitrary
    algorithms on the compressed graph of public development, using various
    mapping techniques to access properties at the node and edge levels.
    We then leverage this infrastructure to study the topology of the entire
    graph, looking at both its local properties and the way software projects
    are organized in forks. The in-depth understanding of this structure then
    allows us to experimentally test and evaluate different approaches for
    distributed graph analysis.

\vspace{3mm}

\textbf{Keywords:} empirical software engineering, source code, open source
software, version control system, digital preservation, graph topology, graph
compression

\end{SingleSpace}
\end{minipage} \\


\begin{minipage}[b][0.49\textheight][t]{0.96\textwidth}
\vspace{6mm}

\section*{Résumé}
\begin{SingleSpace}

Software Heritage est une archive de logiciels contenant la plus
grande collection publique de fichiers de code source ainsi que l'historique de
leur développement, sous la forme d'un immense graphe de centaines de milliards
d'arêtes. Dans cette thèse, nous présentons des techniques architecturales pour
rendre ce graphe disponible à des fins de recherche. Nous proposons d'abord
quelques utilitaires pour accéder aux données à un niveau local d'une manière
adaptée à la recherche à petite échelle. Pour effectuer des analyses sur
l'ensemble de l'archive, nous extrayons un graphe de propriétés dans un format
relationnel et évaluons différents systèmes de traitement pour exploiter ces
données. Cette approche est adaptée au traitement de grandes quantités de
données horizontales, mais elle présente des limites inhérentes à la structure
fortement récursive du graphe. Nous proposons d'utiliser la compression de
graphe comme moyen de réduire considérablement la taille du graphe, ce qui nous
permet de le stocker en mémoire vive. Nous développons une bibliothèque pour
exécuter des algorithmes arbitraires sur le graphe compressé, en utilisant
des techniques de mise en correspondance de ses propriétés au
niveau des nœuds et des arêtes. Nous utilisons ensuite cette infrastructure
pour étudier la topologie locale du graphe ainsi que son organisation en forks
de projets. Comprendre cette structure nous permet ensuite
d'évaluer expérimentalement différentes approches d'analyse distribuée.

\vspace{3mm}

\textbf{Mots-clés~:} génie logiciel empirique, code source, logiciel
open-source, système de gestion de version, conservation numérique, topologie
de graphe, compression de graphe

\end{SingleSpace}
\end{minipage}

\clearpage

\chapter*{Extended abstract}

Software Heritage is an ambitious digital preservation initiative that is
amassing unprecedented quantities of software source code from a variety of
origins, keeping track of all their evolution histories as captured by version
control systems (VCS). The Software Heritage archive has grown to be the
largest and most comprehensive corpus of public software artifacts,
encompassing all the major source code hosts (e.g. GitHub, Gitlab, Bitbucket,
Google Code), supporting a variety of VCS (e.g. Git, Mercurial, SVN) and
package managers (Debian, NixOS, PyPI, NPM…).

This data is stored in a canonical format and deduplicated across the entire
archive at the level of software artifacts, which constructs an immense shared
graph of software development, where the nodes are the source files,
directories and commits, and the edges carry the names of the files and the
links between commits and directories.

The availability of this universal software development knowledge base provides
unique opportunities for what is now known as “Big Code” research: querying,
analysing and extracting knowledge both from the contents of the data and from
the structure of the development history graph. Making this data accessible to
researchers could unlock new capabilities in the software mining field. Being
able to run experiments on the entire graph of public software development
could help us gain a deeper understanding of the evolution of software at a
macro level, and the underlying social structures of software projects. It
would also reduce the barriers of entry to empirical studies by lowering the
costs of data collection and eliminating the need for manual data scraping and
retrieval, as well as facilitate replicating studies at larger scales using
comprehensive data samples.

The exploitation of such an unprecedented source code collection poses
significant challenges in terms of availability, data representation, system
architecture and scalability. The graph in the Software Heritage archive
contains more than 20 billion nodes and 200 billion edges and grows faster
every day. At this scale, few off-the-shelf tools can be used for general
purpose data analysis, and new techniques must be built on the state of the art
of large graph processing.

\subsection*{Data availability}

To better understand the kinds of data that need to be made available, we
systematically categorize the functional requirements of researchers in
software mining studies.  We classify the data available in the Software
Heritage archive, including the content of the source code files, the
development history and other kinds of metadata, but also data that can be
derived from the graph: commit diffs, dependency graphs, community graphs, etc.

Our first step towards providing a general purpose platform for data analysis
is to make the data available in a crude but still exploitable format. As this
can gather interest towards using the archive for research purposes, it
allows us to better understand the challenges of exploiting it, by identifying
the main pain points of researchers and limitations of the different formats.

We provide a few utilities to fetch the data at a micro-level: the vault, a
simple tool to gather the transitive closure of a specific software artifact in
the graph and then representing it in an open format; SwhFS, a virtual
filesystem to run common programs on the code stored in the archive without
having to pre-download it; and finally a simple way to download a list of
source code files from a cloud storage service.

To make possible macro analyses on the entire graph, we extract from the
archive a property graph containing all the software development metadata at
the filesystem, history and hosting levels. We represent it in two different
relational formats: a columnar format for scalable cloud processing, and a
format suitable for import in local databases for in-house analysis. We discuss
various considerations related to the dataset: the different formats,
performance, data denormalization and data privacy.
Putting these datasets on cloud platforms allows us to understand the use cases
for which this relational format is suitable, as well as its limitations,
notably the difficulty of running experiments that require graph traversal
algorithms (e.g walking down a commit chain).

\subsection*{Graph representation}

As the graph dataset is an inherently recursive structure, big data analysis
tools that exploit the ``flatness'' of data to scale-out their processing cannot
be used in a similar fashion on problems that are not embarrassingly parallel.
A naive scale-out approach to randomly assign the artifacts to different shards
is not very efficient, as a traversal algorithm will require jumping
between different shards all the time. We investigate existing scale-out
solutions (GraphFrames, Neo4J, AIGraph) and discuss their limitations for
scale-out graph analysis.

Another approach is that of graph compression: using existing techniques for
compressing very large graphs, we can fit the topology of the Software Heritage
graph in memory on a single machine. This allows us to run standard graph
algorithms without having to find a way to parallelize them. The compressed
graph can be used for prototyping purposes, but also as a production service
that can answer simple traversal queries. The thesis discusses the different
compression techniques used to fit the entire graph topology in only $\sim150$
GB, as well as some storage trade-offs for the node and edge metadata on the
property graph.

To let researchers run experiments on a smaller scale, we introduce a way to
extract representative graph subdatasets of any given size from the archive. We
discuss various implementations of ways to obtain a ``view'' of a given part of
the archive, and make some ``teaser'' datasets available using these techniques.

\subsection*{Graph structure}

Understanding the structure of the development graph is a fundamental step
towards scale-out analysis, as it can help us determine how to partition the
data into tightly knit clusters. Moreover, getting a better sense of the
topology of its different layers as well as the various edge cases and outliers
found in it will help us better organize the data for some domain-specific
algorithms.

As a first step, we look at the low-level topological properties of the graph,
by systematically measuring a few key metrics on its different layers:
in-degree and out-degree distributions, connected component sizes, clustering
coefficients, average path lengths. We compare these metrics with other
large-scale graphs generated by human activities (social networks, graph of the
Web, …).

On a macro-level, we seek to understand how the software artifacts are
organized and shared across different repositories by studying the notion of
``forks'' in the archive. By exploiting the fact that projects that are built on
top of others will share some of their development history, we propose an
algorithm to arrange the list of all the repositories in the archive in
``fork cliques'' (clusters of projects that are all forks of each others) and
``fork networks'' (clusters of projects that are transitively linked together by
forking relationships). We compute these structures for the entire Software
Heritage graph and measure their size distributions.

\subsection*{Applications}

Having both a deeper understanding of the underlying structure of the graph as
well as a platform to run experiments makes it possible to try out different
scale-out approaches experimentally. We evaluate how ``sharding'' the data with
respect to different graph orderings (breadth-first order, layered label
propagation order, fork cliques and connected components) can allow for faster
parallel processing for common use cases.

The analysis platform can also be used to generate derived data from the graph.
As a case study, in order to construct an index that can efficiently find the
first occurrence of a given file in a repository, we showcase a tool to extract
a ``filesystem diff'' dataset, containing the deltas between the source trees
of subsequent commits.

Finally, we discuss future research directions and open subjects: supporting
general-purpose graph queries through an expressive query language, providing
more tooling to streamline scale-out analyses, as well as reducing the lag
between the public datasets and the live data through incremental graph
exports.
