\chapter*{Thesis Structure}

The first part of this thesis introduces the key concepts surrounding
software development, software mining, and how the Software Heritage archive
has the potential to enable research in these fields at a larger, almost
universal scale.
\textbf{\Cref{chp:introduction}} introduces the literature of software mining
and how it evolved over time, from the study of small numbers of source code
files to the systematic exploitation of millions of software projects.
\textbf{\Cref{chp:software-heritage}} presents the Software Heritage archive,
the largest and most diverse corpus of software development artifacts to date,
and the ways in which it can be leveraged to achieve universal software mining.
In \textbf{\cref{chp:version-control-systems}}, we construct an abstract model
for software artifacts stored in deduplicated \glspl{VCS}. This model is then
used in \textbf{\cref{chp:swh-model}} to describe the data model underpinning
the Software Heritage archive, a consolidated, deduplicated and canonicalized
repository of all the software artifacts of public software development.
\textbf{\Cref{chp:large-scale-mining}} reviews recent software mining
literature and categorizes their research needs to help narrow the contours of
our roadmap towards enabling universal software mining.

The second part focuses on ways to make the data stored in the
archive immediately available for software mining purposes under different
formats, as a first step towards building a more comprehensive research
platform. \textbf{\Cref{chp:vault}} introduces the Vault, a server-side tool to
assemble on request bundles of artifacts, from single source code directories
to entire repositories. \textbf{\Cref{chp:fuse}} offers a client-side
alternative to explore small quantities of source code: a virtual filesystem
which presents source code directories stored on the remote archive as if they
were on a local filesystem.  For larger-scale research,
\textbf{\cref{chp:graph-dataset}} introduces a property graph dataset which can
be exploited in big data engines and on cloud analysis platforms.

The third part proposes graph compression as a way to exploit the graph of
software development for resource-hungry recursive algorithms in a relatively
inexpensive manner by fitting the graph on a single machine.
\textbf{\Cref{chp:graph-compression}} introduces techniques for graph
compression applied to the graph of software development, and benchmarks the
compression ratio of the graph and its performance on a few graph traversal
algorithms. \textbf{\Cref{chp:graph-metadata}} explains how different kinds of
graph property data can be attached to the nodes and edges of the compressed
graph for exploitation. \textbf{\Cref{chp:subdatasets}} describes ways to
generate and work on smaller subsets of the compressed graph for prototyping
and filtering purposes. \textbf{\Cref{chp:graph-querying}} explores the problem
of graph querying, and discusses the expressivity of different querying methods
to satisfy complex research needs.

The fourth part attempts, using the datasets and frameworks introduced in the
previous parts, to describe the intrinsic structure of the graph of software
development as a way to understand and document its main properties and to
derive research implications for scale-out analysis approaches. In
\textbf{\cref{chp:topology}} we measure low-level topological properties of the
graph: degree distributions, average path lengths, connected components and
clustering coefficient. In \textbf{\cref{chp:forks}} we take a higher-level
view and look at how software projects are organized in clusters of ``forks''
of varying sizes, largely due to collaborative development and code reuse
patterns.

Finally, we conclude by summarizing our main academic contributions in this
work, discussing the ways they contribute to enabling universal software mining
and their implications for empirical software engineering research, and
exploring potential future research directions.
