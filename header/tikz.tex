\usepackage{tikz}

\usetikzlibrary{backgrounds}
\usetikzlibrary{arrows}
\usetikzlibrary{arrows.meta}
\usetikzlibrary{shapes,shapes.geometric,shapes.misc}
\usetikzlibrary{decorations.pathreplacing}
\usetikzlibrary{patterns}
\usetikzlibrary{positioning}

\usetikzlibrary{calc}
\usepackage{etoolbox}

\pgfdeclarelayer{background}
\pgfdeclarelayer{foreground}
\pgfdeclarelayer{sankeydebug}
\pgfsetlayers{background,main,foreground,sankeydebug}

\newif\ifsankeydebug

\newenvironment{sankeydiagram}[1][]{

  \def\sankeyflow##1##2{% sn, en
    \path[sankey fill]
    let
    \p1=(##1.north east),\p2=(##1.south east),
    \n1={atan2(\y1-\y2, \x1-\x2)-90},
    \p3=(##2.north west),\p4=(##2.south west),
    \n2={atan2(\y3-\y4, \x3-\x4)+90}
    in
    (\p1) to[out=\n1,in=\n2] (\p3) --
    (\p4) to[in=\n1,out=\n2] (\p2) -- cycle;
    \draw[sankey draw]
    let
    \p1=(##1.north east),\p2=(##1.south east),
    \n1={atan2(\y1-\y2, \x1-\x2)-90},
    \p3=(##2.north west),\p4=(##2.south west),
    \n2={atan2(\y3-\y4, \x3-\x4)+90}
    in
    (\p1) to[out=\n1,in=\n2] (\p3)
    (\p4) to[in=\n1,out=\n2] (\p2);
  }


  \tikzset{
    sankey tot length/.store in=\sankeytotallen,
    sankey tot quantity/.store in=\sankeytotalqty,
    sankey min radius/.store in=\sankeyminradius,
    sankey arrow length/.store in=\sankeyarrowlen,
    sankey debug/.is if=sankeydebug,
    sankey debug=false,
    sankey flow/.style={
      to path={
        \pgfextra{
          \pgfinterruptpath
          \edef\sankeystart{\tikztostart}
          \edef\sankeytarget{\tikztotarget}
          \sankeyflow{\sankeystart}{\sankeytarget}
          \endpgfinterruptpath
        }
      },
    },
    sankey node/.style={
      inner sep=0,minimum height={sankeyqtytolen(##1)},
      minimum width=0,draw=none,line width=0pt,
    },
    % sankey angle
    sankey angle/.store in=\sankeyangle,
    % sankey default styles
    sankey fill/.style={line width=0pt,fill,white},
    sankey draw/.style={draw=black,line width=.4pt},
  }

  \newcommand\sankeynode[4]{%prop,orientation,name,pos
    \node[sankey node=##1,rotate=##2] (##3) at (##4) {};
    \ifsankeydebug
    \begin{pgfonlayer}{sankeydebug}
      \draw[red,|-|] (##3.north west) -- (##3.south west);
      \pgfmathsetmacro{\len}{sankeyqtytolen(##1)/3}
      \draw[red] (##3.west)
      -- ($(##3.west)!\len pt!90:(##3.south west)$)
      node[font=\tiny,text=black] {##3};
    \end{pgfonlayer}
    \fi
  }

  \newcommand\sankeynodestart[4]{%prop,orientation,name,pos
    \sankeynode{##1}{##2}{##3}{##4}
    \begin{scope}[shift={(##3)},rotate=##2]
      \path[sankey fill]
      (##3.north west) -- ++(-\sankeyarrowlen,0)
      -- ([xshift=-\sankeyarrowlen/6]##3.west)
      -- ([xshift=-\sankeyarrowlen]##3.south west)
      -- (##3.south west) -- cycle;
      \path[sankey draw]
      (##3.north west) -- ++(-\sankeyarrowlen,0)
      -- ([xshift=-\sankeyarrowlen/6]##3.west)
      -- ([xshift=-\sankeyarrowlen]##3.south west)
      -- (##3.south west);
    \end{scope}
  }

  \newcommand\sankeynodeend[4]{%prop,orientation,name,pos
    \sankeynode{##1}{##2}{##3}{##4}
    \begin{scope}[shift={(##3)},rotate=##2]
      \path[sankey fill]
      (##3.north east)
      -- ([xshift=\sankeyarrowlen]##3.east)
      -- (##3.south west) -- cycle;
      \path[sankey draw]
      (##3.north east)
      -- ([xshift=\sankeyarrowlen]##3.east)
      -- (##3.south west);
    \end{scope}
  }

  \newcommand\sankeyadvance[3][]{%newname,name,distance
    \edef\name{##2}
    \ifstrempty{##1}{
      \def\newname{##2}
      \edef\name{##2-old}
      \path [late options={name=##2,alias=\name}];
    }{
      \def\newname{##1}
    }
    \path
    let
    % sankey node angle
    \p1=(##2.north east),
    \p2=(##2.south east),
    \n1={atan2(\y1-\y2, \x1-\x2)-90},
    % sankey prop
    \p3=($(\p1)-(\p2)$),
    \n2={sankeylentoqty(veclen(\x3,\y3))},
    % next position
    \p4=($(##2.east)!##3!-90:(##2.north east)$)
    in
    \pgfextra{
      \pgfmathsetmacro{\prop}{\n2}
      \pgfinterruptpath
      \sankeynode{\prop}{\n1}{\newname}{\p4}
      \path (\name) to[sankey flow] (\newname);
      \endpgfinterruptpath
    };
  }

  \newcommand\sankeyturn[3][]{%newname,name,angle
    \edef\name{##2}
    \ifstrempty{##1}{
      \def\newname{##2}
      \edef\name{##2-old}
      \path [late options={name=##2,alias=\name}];
    }{
      \def\newname{##1}
    }
    \ifnumgreater{##3}{0}{
      \path
      let
      % sankey node angle
      \p1=(##2.north east),
      \p2=(##2.south east),
      \p3=($(\p1)!-\sankeyminradius!(\p2)$),
      \n1={atan2(\y1-\y2, \x1-\x2)-90},
      % sankey prop
      \p4=($(\p1)-(\p2)$),
      \n2={sankeylentoqty(veclen(\x4,\y4))},
      \p5=(##2.east),
      \p6=($(\p3)!1!##3:(\p5)$)
      in
      \pgfextra{
        \pgfmathsetmacro{\prop}{\n2}
        \pgfinterruptpath
        % \fill[red] (\p3) circle (2pt);
        % \fill[blue](\p6) circle (2pt);
        \sankeynode{\prop}{\n1+##3}{\newname}{\p6}
        \path (\name) to[sankey flow] (\newname);
        \endpgfinterruptpath
      };
    }{
      \path
      let
      % sankey node angle
      \p1=(##2.south east),
      \p2=(##2.north east),
      \p3=($(\p1)!-\sankeyminradius!(\p2)$),
      \n1={atan2(\y1-\y2, \x1-\x2)+90},
      % sankey prop
      \p4=($(\p1)-(\p2)$),
      \n2={sankeylentoqty(veclen(\x4,\y4))},
      \p5=(##2.east),
      \p6=($(\p3)!1!##3:(\p5)$)
      in
      \pgfextra{
        \pgfmathsetmacro{\prop}{\n2}
        \pgfinterruptpath
        % \fill[red] (\p3) circle (2pt);
        % \fill[blue](\p6) circle (2pt);
        \sankeynode{\prop}{\n1+##3}{\newname}{\p6}
        \path (\name) to[sankey flow] (\newname);
        \endpgfinterruptpath
      };
    }
  }

  \newcommand\sankeyfork[2]{%name,list of forks
    \def\name{##1}
    \def\listofforks{##2}
    \xdef\sankeytot{0}
    \path 
    let
    % sankey node angle
    \p1=(\name.north east),
    \p2=(\name.south east),
    \n1={atan2(\y1-\y2, \x1-\x2)-90},
    % sankey prop
    \p4=($(\p1)-(\p2)$),
    \n2={sankeylentoqty(veclen(\x4,\y4))}
    in
    \pgfextra{
      \pgfmathsetmacro{\iprop}{\n2}
    }
    \foreach \prop/\name[count=\c] in \listofforks {
      let
      \p{start \name}=($(\p1)!\sankeytot/\iprop!(\p2)$),
      \n{nexttot}={\sankeytot+\prop},
      \p{end \name}=($(\p1)!\n{nexttot}/\iprop!(\p2)$),
      \p{mid \name}=($(\p{start \name})!.5!(\p{end \name})$)
      in
      \pgfextra{
        \xdef\sankeytot{\n{nexttot}}
        \pgfinterruptpath
        \sankeynode{\prop}{\n1}{\name}{\p{mid \name}}
        \endpgfinterruptpath
      }
    }
    \pgfextra{
      \pgfmathsetmacro{\diff}{abs(\iprop-\sankeytot)}
      \pgfmathtruncatemacro{\finish}{\diff<0.01?1:0}
      \ifnumequal{\finish}{1}{}{
        \message{*** Warning: bad sankey fork (maybe)...}
        \message{\iprop-\sankeytot}
      }
    };
  }

  \tikzset{
    % default values,
    declare function={
      sankeyqtytolen(\qty)=\qty/\sankeytotalqty*\sankeytotallen;
      sankeylentoqty(\len)=\len/\sankeytotallen*\sankeytotalqty;
    },
    sankey tot length=100pt,
    sankey tot quantity=100,
    sankey min radius=30pt,%
    sankey arrow length=10pt,%
    % user values
    #1}
}{
}


\makeatletter
\pgfkeys{/tikz/.cd,
  tab height/.initial=3pt,
  tab width/.initial=10pt,
  tab slope/.initial=1.5pt,
  cover xoff/.initial=5pt,
  cover yoff/.initial=2pt,
}
%% Note: use '\setlength{\pgf@xd}{\pgf@xb}' rather than '\pgf@xd=\pgf@xb'
%% ??? for some reason \pgf@ytt etc. didn't work, but \pgf@ytt does
\newlength\pgf@ytt
\newlength\pgf@xe
\newlength\pgf@xf
\newlength\pgf@xg
\newlength\pgf@yg
\newlength\pgf@xh
%% this didn't work
\newlength\pgf@xo  %% note: \pgf@xx is pre-defined by pgf to mean something

%% from 95.2 Communicating with the Basic Layer via Macros
% \newdimen\pgf@xta \newdimen\pgf@xtb \newdimen\pgf@ytt
% \newdimen\pgf@xoa \newdimen\pgf@xob \newdimen\pgf@yoo
\pgfdeclareshape{document}{ % or 'directory'
  %% this is nearly a rectangle
  \inheritsavedanchors[from=rectangle]
  \inheritanchorborder[from=rectangle]
  \inheritanchor[from=rectangle]{center}
  \inheritanchor[from=rectangle]{north}
  \inheritanchor[from=rectangle]{south}

  \savedmacro\tabheight{%
    \edef\tabheight{\pgfkeysvalueof{/tikz/tab height}}%
  }
  \savedmacro\tabwidth{%
    \edef\tabwidth{\pgfkeysvalueof{/tikz/tab width}}%
  }
  \savedmacro\tabslope{%
    \edef\tabslope{\pgfkeysvalueof{/tikz/tab slope}}%
  }
  \savedmacro\coverxoff{%
    \edef\coverxoff{\pgfkeysvalueof{/tikz/cover xoff}}%
  }
  \savedmacro\coveryoff{%
    \edef\coveryoff{\pgfkeysvalueof{/tikz/cover yoff}}%
  }
  %% this didn't work; probably needs \pgf@process somewhere...
  % \setlength{\pgf@xo}{\coverxoff}
  % \setlength{\pgf@xo}{0.5\pgf@xo}

  %% this didn't work either
  % \setlength{\pgf@xo}{2pt}
  % \advance\pgf@xo by \coverxoff\relax

  %% this didn't work either
  % \pgf@xx=0pt
  % \advance\pgf@xx by \coverxoff\relax
  % \pgf@xx=0.5\pgf@xx

  %% fallback, so this at least compile
  %% but moves text, because \pgf@xx is sed for node contents positioning
  % \pgf@xx=2pt % this should be 0.5 of \coverxoff

  % \inheritanchor[from=rectangle]{west}
  \anchor{west}{
    \pgf@process{\northeast}%
    \pgf@ya=.5\pgf@y%
    \pgf@process{\southwest}%
    \pgf@y=.5\pgf@y%
    % \advance\pgf@y by  \pgf@ya%
    % \advance\pgf@x by -2pt%
  }
  % \inheritanchor[from=rectangle]{east}
  \anchor{east}{%
    \pgf@process{\southwest}%
    \pgf@ya=.5\pgf@y%
    \pgf@process{\northeast}%
    \pgf@y=.5\pgf@y%
    % \advance\pgf@y by  \pgf@ya%
    % \advance\pgf@x by  2pt%
  }
  \backgroundpath{% this is new
    %% store lower left in xa/ya and upper right in xb/yb
    \southwest \pgf@xa=\pgf@x \pgf@ya=\pgf@y
    \northeast \pgf@xb=\pgf@x \pgf@yb=\pgf@y
    %% correction due to open front cover
    % \advance\pgf@xa by -2pt
    % \advance\pgf@xb by -2pt
    %%
    %% compute edges of the "divider tab" in the back cover
    %%
    \setlength{\pgf@ytt}{\pgf@yb}
    \advance\pgf@ytt by \tabheight
    \setlength{\pgf@xe}{\pgf@xa}
    \advance\pgf@xe by \tabwidth % or make it dependent on box width
    \setlength{\pgf@xf}{\pgf@xe}
    \advance\pgf@xf by \tabslope % or make it dependent on divider height
    %%
    %% compute edges of second cover of a folder ("open" cover)
    %%
    \setlength{\pgf@xg}{\pgf@xa}
    \advance\pgf@xg by \coverxoff % perhaps as an angle of open
    \setlength{\pgf@xh}{\pgf@xb}
    \advance\pgf@xh by \coverxoff
    \setlength{\pgf@yg}{\pgf@yb}
    \advance\pgf@yg by-\coveryoff % perhaps as an angle of open
    %%
    %% construct main path
    \pgfpathmoveto{\pgfpoint{\pgf@xa}{\pgf@ya}}
    \pgfpathlineto{\pgfpoint{\pgf@xa}{\pgf@ytt}}
    \pgfpathlineto{\pgfpoint{\pgf@xe}{\pgf@ytt}}
    \pgfpathlineto{\pgfpoint{\pgf@xf}{\pgf@yb}}
    \pgfpathlineto{\pgfpoint{\pgf@xb}{\pgf@yb}}
    \pgfpathlineto{\pgfpoint{\pgf@xb}{\pgf@ya}} % break for open cover
    \pgfpathmoveto{\pgfpoint{\pgf@xb}{\pgf@ya}}
    % \pfgpathlineto{\pgfpoint{\pgf@xa}{\pgf@ya}} % ??? causes error ???
    \pgfpathclose
    %% construct secondary path
    \pgfpathmoveto{\pgfpoint{\pgf@xa}{\pgf@ya}}
    % \pgfpathlineto{\pgfpoint{\pgf@xg}{\pgf@yg}}
    % \pgfpathlineto{\pgfpoint{\pgf@xh}{\pgf@yg}}
    \pgfpathlineto{\pgfpoint{\pgf@xb}{\pgf@ya}}
    \pgfpathclose
  }
}
\makeatother

\input{tikz/swh.tikzstyles}


% this style is applied by default to any tikzpicture included via \tikzfig
\tikzstyle{tikzfig}=[baseline=-0.25em,scale=0.5]

% these are dummy properties used by TikZiT, but ignored by LaTex
\pgfkeys{/tikz/tikzit fill/.initial=0}
\pgfkeys{/tikz/tikzit draw/.initial=0}
\pgfkeys{/tikz/tikzit shape/.initial=0}
\pgfkeys{/tikz/tikzit category/.initial=0}

% standard layers used in .tikz files
\pgfdeclarelayer{edgelayer}
\pgfdeclarelayer{nodelayer}
\pgfsetlayers{background,edgelayer,nodelayer,main}

% style for blank nodes
\tikzstyle{none}=[inner sep=0mm]

% include a .tikz file
\newcommand{\tikzfig}[1]{%
{\tikzstyle{every picture}=[tikzfig]
\IfFileExists{#1.tikz}
  {\input{#1.tikz}}
  {%
    \IfFileExists{./figures/#1.tikz}
      {\input{./figures/#1.tikz}}
      {\tikz[baseline=-0.5em]{\node[draw=red,font=\color{red},fill=red!10!white] {\textit{#1}};}}%
  }}%
}

% the same as \tikzfig, but in a {center} environment
\newcommand{\ctikzfig}[1]{%
\begin{center}\rm
  \tikzfig{#1}
\end{center}}

% fix strange self-loops, which are PGF/TikZ default
\tikzstyle{every loop}=[]
