\usepackage{tikz}

\usetikzlibrary{backgrounds}
\usetikzlibrary{arrows}
\usetikzlibrary{arrows.meta}
\usetikzlibrary{shapes,shapes.geometric,shapes.misc}

\makeatletter
\pgfkeys{/tikz/.cd,
  tab height/.initial=3pt,
  tab width/.initial=10pt,
  tab slope/.initial=1.5pt,
  cover xoff/.initial=5pt,
  cover yoff/.initial=2pt,
}
%% Note: use '\setlength{\pgf@xd}{\pgf@xb}' rather than '\pgf@xd=\pgf@xb'
%% ??? for some reason \pgf@ytt etc. didn't work, but \pgf@ytt does
\newlength\pgf@ytt
\newlength\pgf@xe
\newlength\pgf@xf
\newlength\pgf@xg
\newlength\pgf@yg
\newlength\pgf@xh
%% this didn't work
\newlength\pgf@xo  %% note: \pgf@xx is pre-defined by pgf to mean something

%% from 95.2 Communicating with the Basic Layer via Macros
% \newdimen\pgf@xta \newdimen\pgf@xtb \newdimen\pgf@ytt
% \newdimen\pgf@xoa \newdimen\pgf@xob \newdimen\pgf@yoo
\pgfdeclareshape{document}{ % or 'directory'
  %% this is nearly a rectangle
  \inheritsavedanchors[from=rectangle]
  \inheritanchorborder[from=rectangle]
  \inheritanchor[from=rectangle]{center}
  \inheritanchor[from=rectangle]{north}
  \inheritanchor[from=rectangle]{south}

  \savedmacro\tabheight{%
    \edef\tabheight{\pgfkeysvalueof{/tikz/tab height}}%
  }
  \savedmacro\tabwidth{%
    \edef\tabwidth{\pgfkeysvalueof{/tikz/tab width}}%
  }
  \savedmacro\tabslope{%
    \edef\tabslope{\pgfkeysvalueof{/tikz/tab slope}}%
  }
  \savedmacro\coverxoff{%
    \edef\coverxoff{\pgfkeysvalueof{/tikz/cover xoff}}%
  }
  \savedmacro\coveryoff{%
    \edef\coveryoff{\pgfkeysvalueof{/tikz/cover yoff}}%
  }
  %% this didn't work; probably needs \pgf@process somewhere...
  % \setlength{\pgf@xo}{\coverxoff}
  % \setlength{\pgf@xo}{0.5\pgf@xo}

  %% this didn't work either
  % \setlength{\pgf@xo}{2pt}
  % \advance\pgf@xo by \coverxoff\relax

  %% this didn't work either
  % \pgf@xx=0pt
  % \advance\pgf@xx by \coverxoff\relax
  % \pgf@xx=0.5\pgf@xx

  %% fallback, so this at least compile
  %% but moves text, because \pgf@xx is sed for node contents positioning
  % \pgf@xx=2pt % this should be 0.5 of \coverxoff

  % \inheritanchor[from=rectangle]{west}
  \anchor{west}{
    \pgf@process{\northeast}%
    \pgf@ya=.5\pgf@y%
    \pgf@process{\southwest}%
    \pgf@y=.5\pgf@y%
    % \advance\pgf@y by  \pgf@ya%
    % \advance\pgf@x by -2pt%
  }
  % \inheritanchor[from=rectangle]{east}
  \anchor{east}{%
    \pgf@process{\southwest}%
    \pgf@ya=.5\pgf@y%
    \pgf@process{\northeast}%
    \pgf@y=.5\pgf@y%
    % \advance\pgf@y by  \pgf@ya%
    % \advance\pgf@x by  2pt%
  }
  \backgroundpath{% this is new
    %% store lower left in xa/ya and upper right in xb/yb
    \southwest \pgf@xa=\pgf@x \pgf@ya=\pgf@y
    \northeast \pgf@xb=\pgf@x \pgf@yb=\pgf@y
    %% correction due to open front cover
    % \advance\pgf@xa by -2pt
    % \advance\pgf@xb by -2pt
    %%
    %% compute edges of the "divider tab" in the back cover
    %%
    \setlength{\pgf@ytt}{\pgf@yb}
    \advance\pgf@ytt by \tabheight
    \setlength{\pgf@xe}{\pgf@xa}
    \advance\pgf@xe by \tabwidth % or make it dependent on box width
    \setlength{\pgf@xf}{\pgf@xe}
    \advance\pgf@xf by \tabslope % or make it dependent on divider height
    %%
    %% compute edges of second cover of a folder ("open" cover)
    %%
    \setlength{\pgf@xg}{\pgf@xa}
    \advance\pgf@xg by \coverxoff % perhaps as an angle of open
    \setlength{\pgf@xh}{\pgf@xb}
    \advance\pgf@xh by \coverxoff
    \setlength{\pgf@yg}{\pgf@yb}
    \advance\pgf@yg by-\coveryoff % perhaps as an angle of open
    %%
    %% construct main path
    \pgfpathmoveto{\pgfpoint{\pgf@xa}{\pgf@ya}}
    \pgfpathlineto{\pgfpoint{\pgf@xa}{\pgf@ytt}}
    \pgfpathlineto{\pgfpoint{\pgf@xe}{\pgf@ytt}}
    \pgfpathlineto{\pgfpoint{\pgf@xf}{\pgf@yb}}
    \pgfpathlineto{\pgfpoint{\pgf@xb}{\pgf@yb}}
    \pgfpathlineto{\pgfpoint{\pgf@xb}{\pgf@ya}} % break for open cover
    \pgfpathmoveto{\pgfpoint{\pgf@xb}{\pgf@ya}}
    % \pfgpathlineto{\pgfpoint{\pgf@xa}{\pgf@ya}} % ??? causes error ???
    \pgfpathclose
    %% construct secondary path
    \pgfpathmoveto{\pgfpoint{\pgf@xa}{\pgf@ya}}
    % \pgfpathlineto{\pgfpoint{\pgf@xg}{\pgf@yg}}
    % \pgfpathlineto{\pgfpoint{\pgf@xh}{\pgf@yg}}
    \pgfpathlineto{\pgfpoint{\pgf@xb}{\pgf@ya}}
    \pgfpathclose
  }
}
\makeatother

\input{tikz/swh.tikzstyles}


% this style is applied by default to any tikzpicture included via \tikzfig
\tikzstyle{tikzfig}=[baseline=-0.25em,scale=0.5]

% these are dummy properties used by TikZiT, but ignored by LaTex
\pgfkeys{/tikz/tikzit fill/.initial=0}
\pgfkeys{/tikz/tikzit draw/.initial=0}
\pgfkeys{/tikz/tikzit shape/.initial=0}
\pgfkeys{/tikz/tikzit category/.initial=0}

% standard layers used in .tikz files
\pgfdeclarelayer{edgelayer}
\pgfdeclarelayer{nodelayer}
\pgfsetlayers{background,edgelayer,nodelayer,main}

% style for blank nodes
\tikzstyle{none}=[inner sep=0mm]

% include a .tikz file
\newcommand{\tikzfig}[1]{%
{\tikzstyle{every picture}=[tikzfig]
\IfFileExists{#1.tikz}
  {\input{#1.tikz}}
  {%
    \IfFileExists{./figures/#1.tikz}
      {\input{./figures/#1.tikz}}
      {\tikz[baseline=-0.5em]{\node[draw=red,font=\color{red},fill=red!10!white] {\textit{#1}};}}%
  }}%
}

% the same as \tikzfig, but in a {center} environment
\newcommand{\ctikzfig}[1]{%
\begin{center}\rm
  \tikzfig{#1}
\end{center}}

% fix strange self-loops, which are PGF/TikZ default
\tikzstyle{every loop}=[]
